\documentclass[12pt]{article}
\usepackage{amsmath, amssymb}
\usepackage{natbib}
\usepackage{geometry}
\geometry{margin=1in}
\setlength{\parindent}{0pt}
\setlength{\parskip}{6pt}

\title{AI, Teamwork, and the Dynamics of Skill Formation in the Workplace\\
\normalsize Extended Abstract (8--10 pages)}
\author{---}
\date{}

\begin{document}
\maketitle

\section{Background and Motivation}

Generative artificial intelligence (AI) is reshaping production tasks in knowledge‐intensive industries.  Many firms have begun \emph{replacing} entry-level roles with AI systems that draft contracts, write code, and prepare marketing copy \citep{Appelo2025}.  While seniors paired with AI enjoy large productivity gains \citep{BrynjolfssonLiRaymond2023}, economists have raised concern that eliminating junior positions today could reduce the supply of future experts, because juniors typically acquire tacit knowledge through mentorship \citep{Hess2023,Muehlemann2024}.  This paper develops a stylised model to study that dynamic trade-off.  We ask: under what conditions can AI raise long-run output, and when might it lower it by breaking the \emph{career ladder} that replenishes senior talent?

The model extends the classic two–skill team framework of \citet{Garicano2000} by adding (i) an AI ``worker’’ with costless labour and (ii) a Poisson learning technology through which mentored juniors become seniors.  We connect the resulting efficiency and rent-sharing outcomes to recent theory on excessive automation \citep{AcemogluRestrepo2018,Acemoglu2024}, and to the dynamic competition logic in \citet{BueraBeraja2025}.

\section{Model Setup}

Problems have difficulty $Z\sim U(0,1)$.  A worker of skill $z$ solves any problem with $Z<z$ for sure.

\begin{itemize}
    \item \textbf{Juniors:} skill $z_{0}\in(0,1)$.
    \item \textbf{Seniors:} skill $z_{1}>z_{0}$.
\end{itemize}

Solo output per unit time equals skill: a junior earns $w_{0}=z_{0}$, a senior $w_{1}=z_{1}$.

\paragraph{Team production.}  One senior supervises $n_{0}$ juniors.  Each unsolved task passed upward costs $h<1$ units of the senior’s time.  Capacity implies
\[
n_{0} = \frac{1}{h(1-z_{0})}.
\]
Team output per senior is
\[
Q_{\text{team}} = \frac{z_{1}}{h(1-z_{0})},
\]
so the marginal return to senior skill equals $1/[h(1-z_{0})]>1$.

\paragraph{Static labour‐market equilibrium.}  With supplies $(L_{1},L_{0})$ there are two regimes.
\begin{enumerate}
    \item \emph{Senior‐scarce} ($L_{0}>n_{0}L_{1}$): $w_{0}=z_{0}$ and 
          $w_{1}=(z_{1}-z_{0})/h(1-z_{0})$.
    \item \emph{Junior‐scarce} ($L_{0}<n_{0}L_{1}$): $w_{1}=z_{1}$ and
          $w_{0}=z_{1}[1-h(1-z_{0})]$.
\end{enumerate}
Inequality ($w_{1}/w_{0}$) rises with $z_{1}$ and falls with $z_{0}$; its response to lower $h$ is regime dependent.

\section{Introducing AI}

AI behaves like a cost-free junior with skill $z_{A}$ and communication cost $h_{A}$.  
Output per senior with AI only is 
\[
Q_{\text{AI}}=\frac{z_{1}}{h_{A}(1-z_{A})}.
\]
A senior replaces juniors iff
\[
\frac{z_{1}}{h_{A}(1-z_{A})} \;>\; \frac{z_{1}-z_{0}}{h(1-z_{0})}.  \tag{Adoption}
\]
Because no wage is paid to AI, the inequality holds even when $(z_{A},h_{A})=(z_{0},h)$.
Static GDP rises but juniors are pushed into solo work, widening wage gaps in the senior-scarce regime.

\section{Learning Dynamics}

Population is constant ($L$).  Birth rate $\delta$, death rate $\delta$.
A fraction $\phi$ of newborns are seniors; the rest are juniors.
Juniors on teams become seniors with Poisson rate $\lambda$.

In steady state (still senior-scarce) the senior share is
\[
\frac{L_{1}}{L} = \frac{\phi}{1-\lambda/[\delta h(1-z_{0})]} \;>\;\phi.
\]
Per-capita output without AI is
\[
\frac{Y_{\text{no AI}}}{L} \;=\; z_{0}+\phi\frac{z_{1}-z_{0}[1+h(1-z_{0})]}
                                         {h(1-z_{0})-\lambda/\delta}.
\]

With AI (all juniors solo, no learning), the senior share stays $\phi$ and
\[
\frac{Y_{\text{AI}}}{L} \;=\; \phi\frac{z_{1}}{h_{A}(1-z_{A})} + (1-\phi)z_{0}.
\]

AI lowers long-run GDP iff
\[
\frac{z_{1}}{h_{A}(1-z_{A})} - z_{0}  
  \;<\;
\frac{z_{1}-z_{0}}{h(1-z_{0})-\lambda/\delta}
  - 
\frac{z_{0}}{1-\lambda/[\delta h(1-z_{0})]}.
\]
A larger $\lambda$ raises the right-hand side; learning externalities can therefore outweigh AI’s static gain.

\section{Rent Sharing \& Junior Self-Funding}

If juniors anticipate promotion, they accept a lower current wage:
\[
w_{0}^{\text{team}} = z_{0} - (\lambda/\delta)(w_{1}-z_{0}).
\]
Substituting into firm surplus yields a lower equilibrium $w_{1}$, dampening seniors’ private incentive to adopt AI, but \emph{not} removing the social‐learning externality.

\section{Policy Implications}

The model mirrors the dynamic inefficiency logic of \citet{BueraBeraja2025}: privately optimal AI adoption can be socially excessive when learning externalities matter.  Suggested interventions include:
\begin{itemize}
    \item Neutral tax treatment of capital versus labour to curb automation bias \citep{AcemogluEtAl2023}.
    \item Training subsidies or apprenticeship mandates to maintain $\lambda$ \citep{Muehlemann2024}.
    \item Promoting AI designs that \emph{complement} juniors, raising $z_{0}$ and possibly $\lambda$ \citep{BrynjolfssonLiRaymond2023}.
\end{itemize}

\section{Conclusion}

AI can boost short-run productivity but may erode the learning channel that creates future experts.  When mentoring externalities are strong, laissez-faire adoption can reduce long-run output and widen inequality.  Balancing current gains with future human-capital formation is crucial for \emph{human-centric AI} policy \citep{AcemogluEtAl2023}.

\begin{thebibliography}{}

\bibitem[Acemoglu(2024)]{Acemoglu2024}
Acemoglu,~D.\ (2024).
\newblock \emph{The simple macroeconomics of AI}.
\newblock Unpublished working paper.

\bibitem[Acemoglu \& Restrepo(2018)]{AcemogluRestrepo2018}
Acemoglu,~D., \& Restrepo,~P.\ (2018).
\newblock The race between man and machine: implications of technology for growth, factor shares, and employment.
\newblock \emph{American Economic Review}, 108(6), 1488--1542.

\bibitem[Acemoglu, Johnson, {et~al.}(2023)]{AcemogluEtAl2023}
Acemoglu,~D., Johnson,~S., et~al.\ (2023).
\newblock \emph{Human‐centric AI}.
\newblock CEPR Policy Insight~124.

\bibitem[Appelo(2025)]{Appelo2025}
Appelo,~J.\ (2025, March~26).
\newblock AI wrecks the corporate career ladder.  Give juniors the steering wheel!
\newblock \emph{Medium}.  

\bibitem[Brynjolfsson, Li, \& Raymond(2023)]{BrynjolfssonLiRaymond2023}
Brynjolfsson,~E., Li,~D., \& Raymond,~L.\ (2023).
\newblock \emph{Generative AI at work}.
\newblock (NBER Working Paper No.~31161, rev.\ November 2023).

\bibitem[Buera \& Beraja(2025)]{BueraBeraja2025}
Buera,~F.~J., \& Beraja,~M.\ (2025).
\newblock \emph{The life-cycle of concentrated industries}.
\newblock Manuscript.

\bibitem[Garicano(2000)]{Garicano2000}
Garicano,~L.\ (2000).
\newblock Hierarchies and the organization of knowledge in production.
\newblock \emph{Journal of Political Economy}, 108(5), 874--904.

\bibitem[Hess, Janssen, \& Leber(2023)]{Hess2023}
Hess,~P., Janssen,~S., \& Leber,~U.\ (2023).
\newblock The effect of automation technology on workers’ training participation.
\newblock \emph{Economics of Education Review}, 96, 102438.

\bibitem[Lucas(1988)]{Lucas1988}
Lucas,~R.~E.\ (1988).
\newblock On the mechanics of economic development.
\newblock \emph{Journal of Monetary Economics}, 22(1), 3--42.

\bibitem[Muehlemann(2024)]{Muehlemann2024}
Muehlemann,~S.\ (2024).
\newblock \emph{AI adoption and workplace training} (IZA Discussion Paper No.~17367).

\bibitem[Wang \& Wong(2025)]{WangWong2025}
Wang,~P., \& Wong,~T.~N.\ (2025).
\newblock Artificial intelligence and technological unemployment.
\newblock (NBER Working Paper No.~33867).

\end{thebibliography}

\end{document}
